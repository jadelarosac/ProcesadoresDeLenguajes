Descripción del lenguaje:

Vamos a desarrollar un lenguaje llamado \textit{BABAD}, de forma que consta de una parte común y una específica.

Como parte común, el lenguaje ha de ser un subconjunto de un lenguaje de programación estructurado, donde los identificadores debe ser declarados antes de ser usados. Los tipos de datos básicos a teneren cuenta son: entero, real, carácter y booleano. Para cada una de estos tipos, se necesitará hacer uso de operaciones, como las aritméticas, relacionales y booleanos.

Tipo de dato Operaciones
entero, real
booleano suma, resta, producto, división, operaciones de relación
and, or, not, xor
• Poseerá la sentencia de asignación para todos los tipos de expresiones.
• Permitirá expresiones aritméticas lógicas.
• Tendrá una sentencia de entrada y otra de salida (se utilizará como dispositivo de entrada el teclado
y de salida la pantalla). Además, la sentencia de entrada deberá permitir leer sobre una lista de iden-
tificadores y la sentencia de salida deberá permitir escribir una lista de expresiones y/o constantes
de tipo cadena. A diferencia de los lenguajes conocidos y usados como referencia, estas sentencias
no representan llamada a subprograma.
• Dispone de las estructuras de control siguientes:
– IF-THEN-ELSE.
– WHILE