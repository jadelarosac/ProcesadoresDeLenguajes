\section{Introducción: breve descripción del lenguaje asignado}

En esta práctica vamos a desarrollar un lenguaje de programación,
que consta de una parte común y una específica. A nuestro grupo se le ha
asignado el código \emph{BABAD}, que es el que determina la parte específica.

Como parte común, el lenguaje ha de ser un subconjunto de un lenguaje de
programación estructurado, donde los identificadores debe ser declarados antes
de ser usados. Los tipos de datos básicos a tener en cuenta son: entero, real,
carácter y booleano. Para cada una de estos tipos, se necesitará hacer uso de
operaciones, como las aritméticas, relacionales y booleanas.

Además, la sentencia de asignación se contempla para todos los tipos
de expresiones, permitiendo sentencias con expresiones aritméticas o lógicas.
Existe una sentencia de entrada y otra de salida (se utiliza como dispositivo
de entrada el teclado y de salida la pantalla). Además, la sentencia de entrada
permite leer sobre una lista de identificadores y la sentencia de salida permite
escribir una lista de expresiones y/o constantes de tipo cadena. A diferencia
de los lenguajes conocidos y usados como referencia, estas sentencias no
representan llamada a subprograma.

Nuestro lenguaje, siguiendo el código asignado, se basará en la sintaxis del
lenguaje C y tendrá las palabras reservadas en castellano. Además,
implementa un tipo concreto de estructura de datos: arrays 1D y 2D,
contemplamos operaciones como el acceso a elemento, producto, suma y resta
elemento a elemento, producto externo (producto de un array por un escalar)
y producto de matrices ($C = A \times B$, sabiendo que si A es una matriz
$m \times n$ y B es matriz $k \times l$ entonces ha de darse necesariamente que
$n = k$ y el resultado producto será una matriz C $m \times l$.

Adicionalmente, tendremos en cuenta las variables de tipo array, que solo
pueden tener elementos de los tipos básicos. La forma de implementar
subprogramas es implementar funciones. y además añadiremos como estructura de
control extra el bucle hacer-hasta.
